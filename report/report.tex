\documentclass[a4paper,12pt]{report}

\usepackage{alltt, fancyvrb, url}
\usepackage{graphicx}
\usepackage[utf8]{inputenc}
\usepackage{float}
\usepackage{xcolor}
\usepackage{hyperref}
\usepackage[inkscapelatex=false]{svg}

% Questo commentalo se vuoi scrivere in inglese.
\usepackage[italian]{babel}

\usepackage[italian]{cleveref}

\title{Relazione del progetto\\''coffeBreak''}

\author{Grazia Bochdanovits de Kavna\\
Alessandro Rebosio\\
Filippo Riccioti
}
\date{\today}


\begin{document}

\maketitle

\tableofcontents

\chapter{Analisi}

\section{Descrizione e requisiti}
Il software da noi sviluppato è una riproduzione del celebre videogioco arcade \textit{Donkey Kong}, pubblicato per la prima volta nel 1981 da Nintendo.

Il protagonista è Mario, il cui obiettivo è salvare Pauline, rapita dal gigantesco gorilla Donkey Kong.
%
Il giocatore prende il controllo di Mario e lo guida in un platform a livelli fissi, caratterizzato da strutture complesse
composte da piattaforme, scale e numerosi ostacoli da superare.
%
All'inizio di ogni partita, Mario dispone di tre vite, ogni volta che entra in contatto con un nemico, come barili rotolanti o fiamme, perde una vita e viene riposizionato
all'inizio del livello corrente.
%
La partita continua finché ci sono vite disponibili; una volta esaurite, compare la schermata di \textit{Game Over},
che riporta l'utente al menu principale, da cui è possibile avviare una nuova partita.
%
Il gameplay si basa su una combinazione di tempismo e precisione nei movimenti.
Mario può spostarsi verso sinistra o destra, saltare per evitare i nemici e utilizzare le scale per muoversi verticalmente tra le piattaforme.
L'obiettivo di ogni livello è raggiungere la cima della struttura, dove si trovano Donkey Kong e Pauline.
%
I quattro livelli predefiniti si ripetono ciclicamente ma presentano una difficoltà crescente:
con il progredire del gioco, gli ostacoli si muovono più rapidamente, i nemici diventano meno prevedibili e i percorsi sempre più complessi.
%
Un aspetto centrale del gioco è il sistema di punteggio, che premia il giocatore per varie azioni compiute durante la partita.
I punti vengono assegnati per ogni ostacolo evitato, nemico eliminato o oggetto bonus raccolto, come le monete e il martello,
posizionati in punti chiave del livello.
%
Un ruolo particolare è riservato al martello, un potenziamento temporaneo che consente a Mario di distruggere barili e nemici per alcuni secondi,
offrendo un vantaggio momentaneo e un'opportunità per accumulare punti extra.


\subsection*{Requisiti funzionali}
\begin{itemize}
	\item \textbf{Controlli di gioco:}
	      il giocatore deve poter controllare il personaggio principale tramite input direzionali: camminata (destra/sinistra), salto e salita/discesa scale.
	\item \textbf{Gestione dei livelli e ostacoli:}
	      il sistema deve generare due livelli di gioco in sequenza e gestire la dinamica degli ostacoli, incluso il lancio periodico dei barili da parte di Donkey Kong.
	\item \textbf{Rilevamento e gestione collisioni:}
	      il sistema deve rilevare le collisione tra il personaggio e gli ostacoli/nemici, e applicare le conseguenze previste (es. perdita di vita o bonus).
	\item  \textbf{Sistema di punteggio e condizioni di gioco:}
	      il sistema deve calcolare il punteggio in base a: salti sui barili, raccolta di oggetti bonus ed eliminazione di nemici con il martello.
	      Inoltre, deve rilevare la \textit{condizione di vittoria} quando raggiunge Pauline o se non ci sono più piattaforme distruttibili da rompere, e
	      la \textit{condizione di Game Over} quando esaurisce le vite.
	\item \textbf{Interfaccia utente:}
	      il sistema deve fornire un'interfaccia che mostri in tempo reale il punteggio e  le vite rimanenti.
	\item \textbf{Menù principale:}
	      il sistema deve offrire un menù iniziale con opzioni per avviare una nuova partita e visualizzare i comandi di gioco.
\end{itemize}

\subsection*{Requisiti non funzionali}
\begin{itemize}
	\item Il sistema deve garantire una risposta fluida e immediata ai comandi del giocatore.
	\item Il software deve essere eseguibile su diverse piattaforme desktop, tra cui Windows, macOS e Linux.
	\item Il sistema deve adattarsi correttamente a diverse risoluzioni video standard, mantenendo proporzioni e leggibilità.
\end{itemize}

\newpage
\section{Modello del Dominio}

Il gioco si avvia da un \textbf{menu iniziale}, che consente al giocatore di avviare una nuova partita, visualizzare i cinque punteggi più alti raggiunti nelle sessioni
precedenti, se presenti, oppure uscire dal gioco. Al momento dell'avvio della partita, il gioco carica il \textbf{primo livello} della rotazione ispirata a \textit{Donkey Kong} (Nintendo, 1981), e
posiziona automaticamente il personaggio principale (Jumpman) nella posizione iniziale in basso a sinistra.

La \textbf{partita si sviluppa su due livelli arcade classici}, che si alternano in rotazione ad ogni completamento, ricreando il comportamento dell'originale. Ogni livello
presenta una disposizione fissa di piattaforme, scale, ostacoli e nemici (barili o fiamme), con un obiettivo specifico: raggiungere la cima dello schermo evitando gli ostacoli e
salvare la damigella in pericolo.

Il personaggio è controllabile tramite input da tastiera e può muoversi lateralmente, salire scale e saltare. Il gioco gestisce le collisioni con gli elementi di
gioco (piattaforme, scale, ostacoli e oggetti bonus) permettendo un'esperienza coerente con l'originale.

La visuale è statica: l'intero livello è visibile in una singola schermata. Tuttavia il personaggio se viene colpito da un ostacolo, si attiva la condizione
di \textbf{game over} o di perdita di una vita, mostrando il punteggio accumulato nella sessione in corso e confrontandolo con il record precedente.

Durante la partita, il \textbf{punteggio} è visibile nella parte superiore dello schermo e si aggiorna dinamicamente in base alle azioni
del giocatore (es. raccolta bonus, tempo residuo). È anche possibile mettere il gioco in \textbf{pausa} e riprendere la partita in qualsiasi momento.

Una volta terminata la partita (completamento dei livelli o esaurimento delle vite), viene mostrata una schermata di fine con l'opzione per
tornare al menù principale, avviare una nuova partita o uscire.

\begin{figure}[H]
	\centering{}
	\includesvg[width=1\textwidth,pretex=\tiny]{svg/Model.svg}
	\caption{Schema UML dell'anilisi iniziale, con le principali relazioni fra le entità del Model.}
	\label{img:model}
\end{figure}

\chapter{Design}
\section{Architettura}
L'architettura del nostro progetto adotta il pattern \textbf{Model-View-\newline Controller}, una scelta consolidata
nello sviluppo di applicazioni interattive e, in particolare, di videogiochi. Questo approccio consente una chiara separazione tra:

\begin{itemize}
	\item \textbf{Model}: gestisce lo stato interno del gioco, comprese le entità (come Mario, Donkey Kong, i nemici, le piattaforme, ecc.), le logiche di gioco, le collisioni e
	      il punteggio. L'organizzazione del codice segue il principio della \textit{Single Responsibility}, assegnando a ciascuna classe un compito ben definito.

	\item \textbf{Controller}: gestisce il \textit{game loop} e riceve gli input effettivi da tastiera, che vengono poi inoltrati al Model per essere elaborati come azioni di gioco. Il Controller
	      si occupa quindi di aggiornare, a ogni frame, sia il Model che la View, coordinando l'interazione tra logica di gioco e presentazione visiva.

	\item \textbf{View}: si occupa di tutte le componenti visuali e delle interfacce di gioco. Disegna lo stato corrente durante i vari \textit{game states} (menu, in-game, pausa, game-over),
	      avvalendosi di un \textit{Render Manager} per il disegno delle entità durante il gameplay.
\end{itemize}

Nel caso specifico di \textit{CoffeBreak}, ispirato a \textit{Donkey Kong}, questa struttura ha permesso di progettare un'architettura modulare, facilmente
estendibile e manutenibile. Inoltre, l'adozione del pattern MVC ha facilitato il testing delle singole componenti e reso possibile il lavoro in parallelo tra i membri
del team, migliorando l'efficienza della collaborazione.
\begin{figure}[H]
	\centering{}
	\includesvg[width=1\textwidth,pretex=\tiny]{svg/MVC.svg}
	\caption{Schema UML dell'anilisi iniziale, con le principali relazioni fra le entità dell'architettura.}
	\label{img:View}
\end{figure}

\newpage
\section{Design dettagliato}
\subsection{Alessandro Rebosio}
\textbf{Gestione degli stati del gioco}
\begin{description}
	\item[\textbf{Problema}]
	      Nei videogiochi, è comune avere diversi stati di esecuzione, come \textit{menu}, \textit{in-game}, \textit{pausa} e \textit{game-over}. Una gestione non strutturata di questi stati
	      può portare a codice confuso, con numerosi controlli condizionali sparsi e difficoltà nel mantenimento e nell'estensione del comportamento del gioco. Inoltre, diventa complesso isolare
	      la logica specifica di ciascuno stato, con il rischio di introdurre bug quando si modifica o aggiunge un nuovo stato.

	\item[\textbf{Soluzione}]
	      Per affrontare il problema, abbiamo adottato lo \textbf{State Pattern}, che permette di rappresentare ogni stato del gioco come un oggetto distinto, con comportamento e
	      logica propri. Ogni stato implementa una comune interfaccia, garantendo uniformità nell'aggiornamento e nel rendering. Il \texttt{GameModel} mantiene il riferimento allo stato corrente.
	      Questo approccio migliora la modularità del codice, rende più facile aggiungere o modificare stati in futuro e contribuisce a una migliore organizzazione del flusso di esecuzione.
\end{description}
\begin{figure}[H]
	\centering{}
	\includesvg[width=0.9\textwidth,pretex=\tiny]{svg/ModelState.svg}
	\caption{UML dell'applicazione dello State pattern.}
	\label{img:modelstate}
\end{figure}

\noindent
\textbf{Gestione dei collezionabili}
\begin{description}
	\item[\textbf{Problema}]
	      Nel gioco sono presenti diversi tipi di oggetti collezionabili (ad esempio monete, power-up), che condividono comportamenti comuni come la raccolta, l'attivazione di effetti e la rimozione dalla
	      scena. Implementare queste funzionalità separatamente per ogni tipo di collezionabile porta a duplicazione di codice e rende difficile mantenere coerenza e gestire eventuali modifiche comuni.

	\item[\textbf{Soluzione}]
	      Per risolvere questo problema, abbiamo adottato il \textbf{Template Method}, definendo una classe base astratta per i collezionabili che implementa lo scheletro generale del processo di raccolta,
	      lasciando alle sottoclassi il compito di definire i dettagli specifici (ad esempio l'effetto attivato al momento della raccolta). Questo approccio permette di riutilizzare il codice comune e
	      garantisce una struttura chiara e uniforme, facilitando l'estensione con nuovi tipi di collezionabili senza modificare il comportamento di base.
\end{description}
\begin{figure}[H]
	\centering{}
	\includesvg[width=1\textwidth,pretex=\tiny]{svg/Collectible.svg}
	\caption{UML dell'applicazione del pattern Template Method per diversi Collectible.}
	\label{img:collectible}
\end{figure}

\newpage
\noindent
\textbf{Gestione degli input e comandi di gioco}
%TODO put image of command pattern
\begin{description}
	\item[\textbf{Problema}]
	      Nel controllo di un videogioco, la gestione degli input da tastiera può rapidamente diventare complessa e poco flessibile. Un approccio diretto che collega immediatamente ogni input a un'azione
	      specifica porta a diversi problemi: codice difficile da mantenere e modificare, impossibilità di riutilizzare azioni comuni, difficoltà nell'implementazione di funzionalità avanzate come
	      macro o replay, e complessità nella gestione di input contestuali (dove lo stesso tasto può avere significati diversi in base allo stato del gioco).

	\item[\textbf{Soluzione}]
	      Per risolvere questi problemi, abbiamo implementato il \textbf{Command Pattern}, incapsulando ogni azione di gioco in un oggetto comando distinto. Il \texttt{CommandFactory} si occupa di
	      mappare gli input da tastiera ai comandi appropriati, mentre il \texttt{GameController} esegue i comandi senza conoscere i dettagli dell'implementazione. Abbiamo inoltre implementato comandi
	      contestuali (come \texttt{ContextualCommand}) che cambiano comportamento in base allo stato del gioco, permettendo allo stesso tasto di gestire sia la navigazione nei menu che i movimenti
	      in-game. Questo approccio rende il sistema estendibile, testabile e facilita l'implementazione di funzionalità come la gestione della pressione e del rilascio dei tasti.
\end{description}

\subsection{Grazia Bochdanovits de Kavna}
\textbf{Gestione degli stati di Mario}
\begin{description}
	\item[\textbf{Problema}]
	      Il personaggio di Mario deve adattare il suo comportamento in base al suo stato (normale o con il martello), ma un'implementazione basata su flag booleani avrebbe violato il principio Open/Closed,
	      richiedendo modifiche invasive alla classe di Mario per aggiungere nuovi stati. Questo approccio avrebbe inoltre generato una complessa struttura condizionale, rendendo il codice difficile da mantenere e poco scalabile.

	\item[\textbf{Soluzione}]
	      Per risolvere il problema abbiamo adottato lo \textbf{State Pattern}, che ha permesso di definire un'architettura flessibile e modulare. Attravero l'interfaccia \texttt{CharacterState} sono stati
	      incapsulati i comportamenti del personaggio, mentre le classi concrete, come \texttt{NormalState} e \texttt{WithHammerState}, hanno implementato le specifiche logiche di comportamento.
	      La soluzione ha garantito una chiara separazione delle responsabilità, migliorando significativamente la manutenibilità e l'estendibilità del sistema.
\end{description}

\noindent
\textbf{Gestione delle entità}
\begin{description}
	\item[\textbf{Problema}]
	      Il gioco include numerose entità con caratteristiche sia comuni che specifiche. Senza una adeguata astrazione, si sarebbe verificata una pericolosa duplicazione di codice per la gestione degli attributi
	      comuni (posizione, dimensione e velocità), rendendo il sistema difficile da mantenere e soggetto a errori.

	\item[\textbf{Soluzione}]
	      L'adozione del \textbf{Template Method} ha permesso di creare una gerarchia multilivello di classi astratte. La classe base \texttt{AbstractEntity} è stata progettata per incapsulare tutta la logica comune
	      alle entità, mentre classi intermedie come \texttt{AbstractEnemy} o \texttt{AbstractPlatform} hanno catturato i comportamenti condivisi da gruppi specifici di entità.
	      Le classi concrete, a loro volta, hanno potuto specializzare solo gli aspetti veramente unici di ciscuna entità. Questa architettura ha notevolemente migliorato il riuso, semplificando l'estensione del sistema
	      con l'aggiunta di nuove entità.

\end{description}

\noindent
\textbf{Gestione del rendering delle entità}
\begin{description}
	\item[\textbf{Problema}]
	      Il sistema di rendering doveva gestire numerose entità con caratteristiche grafiche diverse. Un aproccio centralizzato avrebbe creato un accoppiamento eccessivo
	      tra la logica di rendering e le implementazioni delle singole entità, rendendo il sistema rigido e difficilmente estendibile.
	      Inoltre, la necessità di gestire operazioni comuni (come il caricamento degli sprite) in modo indipendente, avrebbe portato a una significativa duplicazione di codice
	      violando il principio DRY.

	\item[\textbf{Soluzione}]
	      Abbiamo implementato un'architettura che combina lo \textbf{Strategy Pattern} con un registro di renderer. Ciascuna entità dispone di un renderer dedicato che implementa l'interfaccia \texttt{EntityRender}.
	      Il \texttt{GameRenderManager} utilizza una mappa di registrazione per associare dinamicamente ogni tipo di entità al corrispondente renderer. Questa soluzione riduce l'accoppiamento, facilita l'aggiunta di
	      nuovi renderer e migliora l'efficienza grazie all'inizializzazione unica dei componenti.
\end{description}

\newpage
\subsection{Filippo Ricciotti}

\noindent
\textbf{Gestione dei livelli di gioco (\texttt{GameLevelManager})}
\begin{description}
	\item[\textbf{Problema}]
	      La gestione dei livelli di gioco era implementata in modo rigido, con controlli manuali per determinare il livello corrente e per passare al successivo. Questo approccio non favoriva l'
	      estendibilità e rendeva difficile introdurre nuove logiche relative ai livelli, come eventi specifici o condizioni di completamento personalizzate. Inoltre, il caricamento e l'inizializzazione dei
	      livelli erano distribuiti in più punti del codice, aumentando la complessità.

	\item[\textbf{Soluzione}]
	      Per risolvere il problema, abbiamo strutturato la gestione dei livelli applicando il \textbf{State Pattern}, trattando ogni livello come uno stato distinto del gioco. Il \texttt{GameLevelManager}
	      si occupa esclusivamente della transizione tra stati-livello, centralizzando le logiche di passaggio, caricamento e inizializzazione. Questo consente una gestione più modulare, facilitando
	      l'aggiunta di nuovi livelli.
\end{description}
\begin{figure}[H]
	\centering{}
	\includesvg[width=1\textwidth,pretex=\tiny]{svg/GameLevelManager.svg}
	\caption{UML dell'applicazione del GameLevelManager.}
	\label{img:GameLevelManager}
\end{figure}

\noindent
\textbf{Gestione delle entità di gioco (\texttt{GameEntityManager})}
\begin{description}
	\item[\textbf{Problema}]
	      La gestione delle entità (personaggi, ostacoli, collezionabili) era organizzata tramite liste generiche, senza una chiara separazione tra i diversi tipi di entità. Questo causava
	      difficoltà nel controllo delle collisioni, aggiornamento dello stato e rendering, oltre a rendere il codice più difficile da mantenere e soggetto a errori legati alla gestione dei tipi.

	\item[\textbf{Soluzione}]
	      È stata realizzata la classe \texttt{GameEntityManager}, incaricata di gestire tutte le entità presenti nel gioco. Questa classe fornisce metodi per aggiungere, rimuovere e aggiornare le
	      entità, garantendo un unico punto di accesso e controllo. Ciò favorisce una gestione ordinata e modulare, rendendo più semplice implementare logiche di gioco complesse come le collisioni o
	      le interazioni tra entità.
\end{description}
\begin{figure}[H]
	\centering{}
	\includesvg[width=1\textwidth,pretex=\tiny]{svg/GameEntityManager.svg}
	\caption{UML della gestione modulare delle Entità grazie al GameEntityManager.}
	\label{img:GameEntityManager}
\end{figure}
\newpage
\noindent
\textbf{Gestione delle mappe di gioco (\texttt{GameMapsManager})}
\begin{description}
	\item[\textbf{Problema}]
	      Il caricamento delle mappe e il loro ordinamento venivano gestiti tramite logiche manuali e ripetitive, basate su cicli e controlli espliciti. Questo rendeva il codice meno
	      leggibile e più difficile da estendere con nuovi criteri di ordinamento o formati di mappa differenti.

	\item[\textbf{Soluzione}]
	      Abbiamo applicato il \textbf{Strategy Pattern} per definire criteri di ordinamento intercambiabili tra mappe, separando la logica di confronto da quella di caricamento. In
	      aggiunta, abbiamo sfruttato le \textbf{Stream API} di Java per semplificare le operazioni su collezioni, come il filtro e il conteggio di elementi, migliorando la chiarezza e la concisione
	      del codice. Questo approccio facilita la manutenzione e l'aggiunta di nuove funzionalità legate alle mappe.
\end{description}
\begin{figure}[H]
	\centering{}
	\includesvg[width=1\textwidth,pretex=\tiny]{svg/GameMapsManager.svg}
	\caption{UML dell'applicazione dello Strategy pattern per gestire diverse mappe.}
	\label{img:GameMapsManager}
\end{figure}
\newpage
\noindent
\textbf{Gestione degli stati grafici (\texttt{AbstractViewState} e \texttt{ViewStates})}
\begin{description}
	\item[\textbf{Problema}]
	      Nei diversi \texttt{ViewState} del gioco (come \texttt{MenuViewState}, \newline \texttt{InGameViewState}, \texttt{PauseViewState}, \texttt{GameOverViewState}), la gestione dei comandi dell'utente
	      presentava duplicazioni, specialmente per le azioni comuni come il ritorno al menu o l'uscita dal gioco. Questo aumentava il rischio di incongruenze e complicava l'aggiunta di nuove
	      funzionalità condivise tra gli stati.

	\item[\textbf{Soluzione}]
	      Abbiamo adottato il \textbf{Template Method Pattern} all'interno della classe astratta \texttt{AbstractViewState}, definendo un metodo finale che gestisce i comandi comuni a tutti gli stati,
	      delegando solo i comandi specifici ai sottotipi. In questo modo, la logica ripetuta viene centralizzata, semplificando la manutenzione e migliorando la coerenza tra i diversi stati grafici del gioco.
\end{description}
\begin{figure}[H]
	\centering{}
	\includesvg[width=1\textwidth,pretex=\tiny]{svg/AbstractViewState.svg}
	\caption{UML dell'applicazione delo State Pattern Method per la gestione delle diverse interfacce visive.}
	\label{img:AbstractViewState}
\end{figure}
\newpage
\chapter{Sviluppo}
\section{Testing automatizzato}
In questo progetto abbiamo implementato dei test unitari con JUnit 5 per tutte le classi principali. I test progettati assicurano la verifica automatica delle funzionalità fondamentali del software. Di seguito vengono riportati alcuni esempi di test implementati.

\subsubsection*{Common}
\begin{itemize}
	\item \texttt{TestBoundingBox:} verifica la corretta creazione e il corretto funzionamento delle dimensioni delle entità.
	\item \texttt{TestResourceLoader:} viene verificato il giusto caricamento delle risorse del gioco.
	\item \texttt{TestPosition:} testa la gestione delle posizioni nello spazio di gioco.
	\item \texttt{TestVector:} verifica le operazioni sui vettori.
\end{itemize}

\subsubsection*{Controller}
\begin{itemize}
	\item \texttt{TestInputManager:} controlla la gestione degli input da tastiera.
	\item \texttt{TestGameController:}  testa la logica principale di controllo del gioco.
\end{itemize}

\subsubsection*{Repository}
\begin{itemize}
	\item \texttt{TestScoreRepository:} verifica la corretta gestione della persistenza e del recupero dei punteggi nella repository.
\end{itemize}

\newpage
\subsubsection*{Model}
\begin{itemize}
	\item \texttt{TestMario}: controlla che il movimento e le funzionalità del personaggio principale avvengano correttamente.
	\item \texttt{TestLivesManager}: controlla la gestione delle vite del personaggio principale.
	\item \texttt{TestPlatform}: controlla il comportamento delle piattaforme normal e breackable.
	\item \texttt{TestLadder}: verifica il funzionamento delle scale.
	\item \texttt{TestGameTank}: verifica il funzionamento della tanica d'olio.
	\item \texttt{TestPauline}: testa il comportamento dell'ostaggio.
	\item \texttt{TestDonkeyKong}: verifica la logica del lancio dei barili e il comportamento dell'antagonista.
	\item \texttt{TestFire}: verifica il comportamento delle fiamme.
	\item \texttt{TestCollectible}: testa la raccolta degli oggetti collezionabili (monete, martelli) e il loro effetto.
	\item \texttt{TestEntry:} verifica la corretta gestione delle singole voci della classifica.
	\item \texttt{TestGameLeaderBoard:} verifica la gestione e il salvataggio dei punteggi nella leaderboard.
	\item \texttt{TestScore:} controlla la gestione e il calcolo dei punteggi.
	\item \texttt{TestBonus:} verifica la logica dei bonus di gioco.
\end{itemize}

\newpage
\section{Note di sviluppo}

\subsection{Alessandro Rebosio}
\subsubsection{Progettazione con \texttt{generics}}
Utilizzati i generics per definire interfacce e classi parametrizzate. Permalink
\begin{sloppypar}
	\raggedright
	\url{https://github.com/alessandrorebosio/OOP24-coffeBreak/blob/3a264084eabd86bac19f9978d9c2aba0c1cbf624/src/main/java/it/unibo/coffebreak/api/common/State.java#L14}
\end{sloppypar}

\subsubsection{Utilizzo di \texttt{Stream}}
Utilizzati di frequente, soprattutto per il controllo sulle entità. Permalink di un esempio
\begin{sloppypar}
	\raggedright
	\url{https://github.com/alessandrorebosio/OOP24-coffeBreak/blob/bcbde8130ad334ff018023e0fdb19b8c180575b6/src/main/java/it/unibo/coffebreak/impl/model/physics/GamePhysicsEngine.java#L149C1-L155C20}
\end{sloppypar}

\subsubsection{Utilizzo di lambda expressions}
Utilizzati di frequente, nel caricamento delle risorse. Permalink di un esempio:
\begin{sloppypar}
	\raggedright
	\url{https://github.com/alessandrorebosio/OOP24-coffeBreak/blob/3a264084eabd86bac19f9978d9c2aba0c1cbf624/src/main/java/it/unibo/coffebreak/impl/common/ResourceLoader.java#L99}
\end{sloppypar}

\subsubsection{Gestione degli \texttt{Optional}}
Usato per gestire valori che potrebbere essere assenti. Permalink di un esempio
\begin{sloppypar}
	\raggedright
	\url{https://github.com/alessandrorebosio/OOP24-coffeBreak/blob/3a264084eabd86bac19f9978d9c2aba0c1cbf624/src/main/java/it/unibo/coffebreak/impl/model/states/ingame/InGameModelState.java#L57-L61}
\end{sloppypar}

\newpage
\subsection{Grazia Bochdanovits de Kavna}
\subsubsection{Utilizzo di \texttt{Stream}}
Utilizzati di frequente. Permalink di un esempio
\begin{sloppypar}
	\raggedright
	\url{https://github.com/alessandrorebosio/OOP24-coffeBreak/blob/926d3f1b985c9bba79c5f766ac0070b04fd6185f/src/main/java/it/unibo/coffebreak/impl/view/states/ingame/InGameView.java#L62-L65}
\end{sloppypar}

\subsubsection{Gestione degli \texttt{Optional}}
Usato per gestire valori che potrebbere essere assenti. Permalink di un esempio
\begin{sloppypar}
	\raggedright
	\url{https://github.com/alessandrorebosio/OOP24-coffeBreak/blob/926d3f1b985c9bba79c5f766ac0070b04fd6185f/src/main/java/it/unibo/coffebreak/impl/view/render/GameRenderManager.java#L82-L87}
\end{sloppypar}

\subsubsection{Utilizzo di lambda expressions}
Utilizzati di frequente, soprattuto nel controllo delle collisioni fra entità. Permalink di un esempio:
\begin{sloppypar}
	\raggedright
	\url{https://github.com/alessandrorebosio/OOP24-coffeBreak/blob/926d3f1b985c9bba79c5f766ac0070b04fd6185f/src/main/java/it/unibo/coffebreak/impl/model/entities/mario/Mario.java#L173-L186}
\end{sloppypar}

\newpage
\subsection{Filippo Ricciotti}
\subsubsection{Utilizzo di \texttt{Stream}}
\begin{sloppypar}
	\raggedright
	\url{https://github.com/alessandrorebosio/OOP24-coffeBreak/blob/56f59dcd01020b02589aa88b808de452c07a3f05/src/main/java/it/unibo/coffebreak/impl/model/level/entity/GameEntityManager.java#L148-L159}
\end{sloppypar}

\subsubsection{Gestione degli \texttt{Optional}}
\begin{sloppypar}
	\raggedright
	\url{https://github.com/alessandrorebosio/OOP24-coffeBreak/blob/56f59dcd01020b02589aa88b808de452c07a3f05/src/main/java/it/unibo/coffebreak/impl/model/level/GameLevelManager.java#L122C35-L122C36}
\end{sloppypar}

\subsubsection{Utilizzo di lambda expressions}
\begin{sloppypar}
	\raggedright
	\url{https://github.com/alessandrorebosio/OOP24-coffeBreak/blob/56f59dcd01020b02589aa88b808de452c07a3f05/src/main/java/it/unibo/coffebreak/impl/model/level/entity/GameEntityManager.java#L230}
\end{sloppypar}
\chapter{Commenti finali}
\section{Autovalutazione e lavori futuri}
\subsection{Alessandro Rebosio}
La realizzazione di questo progetto ha rappresentato per me un'esperienza formativa di grande valore. Dopo un'attenta analisi iniziale, ho iniziato subito a implementare le classi principali, scoprendo gradualmente come alcune parti potessero essere migliorate attraverso operazioni di refactoring. Questa esperienza mi ha fatto comprendere profondamente l'utilità dei design pattern per organizzare il lavoro e ridurre la complessità complessiva del sistema.

Essendo il mio primo progetto strutturato di queste dimensioni, ho posto particolare attenzione alla scrittura di codice chiaro, modulare ed espandibile, principi che si sono rivelati fondamentali per facilitare la manutenzione e l'evoluzione del software. Aver lavorato su un progetto concreto mi ha permesso di comprendere meglio tutto l'ambiente di sviluppo: dall'utilizzo avanzato di strumenti come Git per il controllo di versione, alla gestione delle dipendenze con Gradle, fino alle metodologie di testing con JUnit.

All'interno del gruppo ho ricoperto diversi ruoli, concentrandomi principalmente sullo sviluppo del Controller e definendo le basi architetturali per la View, successivamente sviluppata dai miei colleghi. Questa esperienza trasversale mi ha permesso di sperimentare direttamente la sincronizzazione e l'interazione tra le diverse componenti dell'architettura MVC, approfondendo la comprensione delle dinamiche di progettazione in un contesto collaborativo reale.

\subsection{Filippo Ricciotti}
Riflettendo sul progetto svolto, mi rendo conto che avrei potuto dedicare più tempo e attenzione a questo progetto rispetto che ad altri esami, bilanciando meglio le priorità. Questo avrebbe sicuramente alleggerito il carico di lavoro per tutto il gruppo, soprattutto in alcune fasi critiche come l’integrazione tra le varie componenti e la gestione dei dettagli tecnici. Alcune difficoltà incontrate, come la sincronizzazione tra Model, View e Controller avrebbero richiesto meno sforzo complessivo se mi fossi impegnato con maggiore costanza.

Nonostante ciò, considero questa esperienza molto formativa: mi ha dato una prima impressione concreta di come si lavora in ambito aziendale, con l'importanza della collaborazione e della divisione dei compiti. Ho potuto apprezzare l'utilità di strumenti come \texttt{Git} per la gestione del codice condiviso, e l'efficacia di ambienti di sviluppo come \texttt{Visual Studio Code}. Inoltre, i concetti della \texttt{programmazione ad oggetti}, come l'incapsulamento, l'ereditarietà e l'uso dei design pattern, mi sono rimasti impressi e li considero fondamentali per organizzare progetti complessi in modo chiaro e mantenibile.

Questa esperienza mi ha fatto capire l'importanza di una pianificazione precisa e di un coinvolgimento costante, elementi che cercherò di applicare meglio nei progetti futuri.

\section{Difficoltà incontrate e commenti per i docenti}

Questa sezione, \textbf{opzionale}, può essere utilizzata per segnalare ai docenti eventuali problemi o difficoltà incontrate nel corso o nello svolgimento del progetto, può essere vista come una seconda possibilità di valutare il corso (dopo quella offerta dalle rilevazioni della didattica) avendo anche conoscenza delle modalità e delle difficoltà collegate all'esame, cosa impossibile da fare usando le valutazioni in aula per ovvie ragioni.
%
È possibile che alcuni dei commenti forniti vengano utilizzati per migliorare il corso in futuro: sebbene non andrà a vostro beneficio, potreste fare un favore ai vostri futuri colleghi.
%
Ovviamente \textit{il contenuto della sezione non impatterà il voto finale}.

\appendix
\chapter{Guida utente}

\section{Menu Principale}
All'avvio dell'applicazione, l'unico strumento a disposizione sarà la tastiera. Nel menu principale sarà possibile visualizzare la leaderboard e selezionare le opzioni disponibili tramite le frecce \texttt{SU} e \texttt{GIÙ}. Per confermare la selezione si utilizza il tasto \texttt{ENTER}.

\section{Durante il Gioco}
Entrati in gioco, si parte dal primo livello: lo scopo è raggiungere la principessa evitando i nemici e raccogliendo power-up.

Per giocare basta usare la tastiera: ogni \texttt{freccia} muove il personaggio nella rispettiva direzione, la \texttt{barra spaziatrice} fa saltare, mentre le frecce \texttt{SU} e \texttt{GIÙ} funzionano solo in corrispondenza di una scala. In qualsiasi momento, premendo il tasto \texttt{ESC} puoi mettere il gioco in pausa.

\section{Menu di Pausa}
Nel menu di pausa si naviga tra le opzioni disponibili con le frecce \texttt{SU} e \texttt{GIÙ}, e si seleziona l'opzione desiderata con \texttt{ENTER}.

\section{Game Over}
Quando il personaggio perde tutte le vite, si entra nella schermata di Game Over. In questa schermata l'unica azione possibile è premere \texttt{ENTER} per tornare al menu principale.

\chapter{Esercitazioni di laboratorio}
\section{alessandro.rebosio@studio.unibo.it}

\begin{itemize}
	\item Laboratorio 07: \url{https://virtuale.unibo.it/mod/forum/discuss.php?d=177162#p246059}
	\item Laboratorio 09: \url{https://virtuale.unibo.it/mod/forum/discuss.php?d=179154#p248324}
	\item Laboratorio 10: \url{https://virtuale.unibo.it/mod/forum/discuss.php?d=180101#p249805}
	\item Laboratorio 11: \url{https://virtuale.unibo.it/mod/forum/discuss.php?d=181206#p250995}
\end{itemize}


\bibliographystyle{alpha}
\bibliography{report}
\nocite{*}

\end{document}
